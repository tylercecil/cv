\documentclass{tc_cv}

\newcommand{\jobSpace}{\vspace{0.75em}}

\begin{document}
\makeheader{Senior Engineer, Ex-Googler, Pivoting to Climate}

\begin{minipage}[t]{0.3\linewidth}
  \emphtext{Senior Engineer,\\
  \indent\indent known for leadership and technical expertise,
  looking to impacts the environment.
  }

  \vspace{2em}
  \section{Education}
  \company{New Mexico Institute \\ of Technology}\\
  \date{Fall 2011 --- Spring 2016}\vspace{0.25em}\\
  \faGraduationCap \hspace{1ex} B.S. Mathematics\\
  \faGraduationCap \hspace{1ex} B.S. Computer Science

  \vspace{2em}
  \section{Skills}
  \subsection{Languages}
  Python,
  Go,
  SQL,
  R,
  Haskell,
  Rust,
  C/C++,
  Java,
  \ldots

  \subsection{Topics}
  Distributed Systems,
  System Design,
  Data Science,
  Mathematics,
  Test Driven Development,
  \ldots

  \subsection{Tools and Frameworks}
  Unix/Linux,
  Version Control / Git,
  Google Cloud Platform (GCP),
  Jupyter,
  Databasses,
  \ldots

  \subsection{Leadership}
  Team Management,
  Technical Writing,
  Communication,
  Project Planning,
  Prioritization,
  \ldots

  \vspace{2em}
  \section{Personal Projects}
  \textbf{Research Blog}\\
  Maintain \href{https://tylercecil.com}{tylercecil.com}, a personal blog on
  Mathematics, Data Science, and other academic interests.
  \vspace{0.25em}

  \textbf{Interests}\\
  Actively engaged outdoors, rock climbing, backpacking, and bikepacking. When
  inside, I play music, learn languages, and advocate for local cycling policy.

\end{minipage}
\hfill\vline\hfill
\begin{minipage}[t]{0.6\linewidth}
  \section{Employment}
  \begin{job}{Google}
    {April 2020 --- March 2022}
    {Senior Software Engineer, Technical Lead}
    % Wishlist:
    %  - "go-to person" or "technical expert"
    %  - Python & SQL
    %  - Data
    \item Headed team releasing Linux kernels for Google Cloud Platform's
      virtualization hosts, working with low-level kernel and platform code,
      and high-level data analysis and release tools.
    \item Reduced average release cadence from over a year to under a monthly
      by building a bespoke automated release management tool for the unique
      requirements of host-code.
    \item Set cross-team technical direction, writing and tracking OKRs for
      multiple multi-year efforts in kernel virualization space.
    \item Managed junior developers, setting tasks and expectations, as well as
      providing mentorship.
  \end{job}
  \jobSpace
  \begin{job}{Google}
    {June 2016 --- June 2018}
    {Software Engineer}
  \item Created critical tools for GCP's networking layer: notably a scaled
    net-probe framework in Go, ensuring network stability.
  \item Contributed to a project allowing customers to opt-in to GCP's bleeding
    edge, modifying a complex microservice architecture, and working closely
    with users while on-boarding.
  \item Took on general team responsibilities, including interviewing, on-call
    rotation, and planning team OKRs.
  \end{job}
  \jobSpace
  \begin{job}{Institute for Complex \\ Additive Systems Analysis}
    {June 2014 --- April 2016}
    {Software Engineer}
    \item Developed data model for Java apps containing classified info,
      using partial encryption and replacing JVM object serialization.
    \item Introduced continuous integration system, modern version control, and
      Agile methodologies to the institute.
    \item Built requirements and project timeline directly with clients.
  \end{job}
  \jobSpace
  \begin{job}{Los Alamos National Laboratory}
    {Summer 2015}
    {Researcher Intern}
    \item Designed LLVM extension, translating high-level parallel semantics
      in C++ to a low-level IR for HPC applications.
    \item Published findings and related research in internal laboratory
      communications, bringing future developers up to speed.
  \end{job}
  \jobSpace
  \begin{job}{Magdalena Ridge \\ Observatory}
    {Spring 2013 --- June 2014}
    {Instrumentation Engineer}
    \item Implemented control and analysis software in Python for
      \href{https://noirlab.edu/public/programs/kitt-peak-national-observatory/wiyn-35m-telescope/nessi/}{NESSI},
      an astrophysics spectroscopy instrument.
    \item Scope included Linux drivers for imaging and motor hardware, a Python
      GUI, and image analysis tools using SciTools.
    \item Highly multidisciplinary, requiring quick-learning, and communicating
      progress and constraints to non-software audience.
  \end{job}
\end{minipage}

\end{document}
