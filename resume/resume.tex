\documentclass{tc_cv}

\newcommand{\jobSpace}{\vspace{0.75em}}

\begin{document}
\makeheader{Senior Engineer, Ex-Googler, Pivoting to Climate}

\begin{minipage}[t]{0.3\linewidth}
  \emphtext{
    Senior Engineer, \\
    with advanced background in Mathematics,\\
    looking for environmental impact.
  }

  \vspace{1.5em}
  \section{Education}
  \company{New Mexico Institute \\ of Technology}\\
  \date{Fall 2011 --- Spring 2016}\vspace{0.25em}\\
  \faGraduationCap \hspace{1ex} B.S. Mathematics\\
  \faGraduationCap \hspace{1ex} B.S. Computer Science

  \vspace{1.5em}
  \section{Skills}
  \subsection{Languages}
  Python,
  Go,
  Julia,
  Rust,
  C/C++,
  Java,
  Haskell,
  Lisp,
  \ldots

  \vspace{0.25em}
  \subsection{Topics}
  % Clean these up
  Algorithms,
  Mathematics,
  Modeling,
  Data Science,
  Software Architecture,
  Distributed Systems,
  Agile \&
  Test Driven Development,
  \ldots

  \vspace{0.25em}
  \subsection{Tools and Frameworks}
  Unix,
  Git,
  CI / CD,
  %GCP,
  Databases,
  Pandas, Numpy, Scipy,
  Jupyter,
  \ldots

  \vspace{0.25em}
  \subsection{Leadership}
  Team Management,
  Technical Writing,
  Communication,
  Technical Mentorship,
  Cross-functional Work,
  \ldots

  \vspace{1.5em}
  \section{Personal Projects}
  \textbf{Research Blog}\\
  Maintain \href{https://tylercecil.com}{\ul{tylercecil.com}}, a personal
  research blog on Math, Data Science, Language, and other academic interests.

  \vspace{0.5em}
  \textbf{Environmental Activism}\\
  Engaged in political action groups, such as the California Bicycle Coalition
  and the American Climbing Access Fund, advocating for green public policy.

\end{minipage}
\hfill\vline\hfill
\begin{minipage}[t]{0.6\linewidth}
  \section{Employment}
  \begin{job}{Google}
    {April 2020 --- March 2022}
    {Senior Software Engineer, Technical Lead}
    % Wishlist:
    %  - "go-to person" or "technical expert"
    %  - Python & SQL
    %  - Data
    \item Headed team releasing Linux kernels for Google Cloud Platform's
      virtualization hosts, working with low-level kernel and platform code,
      and high-level data analysis and release tools.
    \item Reduced average release cadence from over a year to under a monthly
      by building a bespoke automated release management tool for the unique
      requirements of host-code.
    \item Set cross-team technical direction, writing and tracking OKRs for
      multiple multi-year efforts in kernel virualization space.
    \item Managed junior developers, setting tasks and expectations, as well as
      providing mentorship.
  \end{job}
  \jobSpace
  \begin{job}{Google}
    {June 2016 --- June 2018}
    {Software Engineer}
  \item Created critical tools for GCP's networking layer: notably a scaled
    net-probe framework in Go, ensuring network stability.
  \item Contributed to a project allowing customers to opt-in to GCP's bleeding
    edge, modifying a complex microservice architecture, and working closely
    with users while on-boarding.
  \item Actively engaged with team responsibilities, including interviewing,
    an on-call rotation, and planning team OKRs.
  \end{job}
  \jobSpace
  \begin{job}{Institute for Complex \\ Additive Systems Analysis}
    {June 2014 --- April 2016}
    {Software Engineer}
    \item Developed data model for Java apps containing classified info,
      using partial encryption and replacing JVM object serialization.
    \item Introduced continuous integration system, modern version control, and
      Agile methodologies to the institute.
  \end{job}
  \jobSpace
  \begin{job}{Los Alamos National Laboratory}
    {Summer 2015}
    {Research Intern}
    \item Designed LLVM extension, translating high-level parallel
      semantics in C++ to a low-level IR for HPC applications.
    \item Published findings and related research in internal laboratory
      journal, bringing future developers up to speed.
  \end{job}
  \jobSpace
  \begin{job}{Magdalena Ridge Observatory}
    {March 2013 --- June 2014}
    {Instrumentation Engineer}
    \item Implemented control and analysis software in Python for
      \href{https://noirlab.edu/public/programs/kitt-peak-national-observatory/wiyn-35m-telescope/nessi/}{NESSI},
      an astrophysics spectroscopy instrument.
    \item By providing Linux device drivers for imaging and motor hardware,
      GUI, and image analysis tools using Jupyter and SciTools, enabled an
      end-to-end astronomy workflow.
    \item Navigated highly multidisciplinary workspace, quickly learning new
      fields, while communicating to non-software audience.
  \end{job}
\end{minipage}

\end{document}
